% !TeX document-id = {442e212a-a1ac-42c6-bf9c-a93a91483fa1}
% These are "magic" comments understood by the "TeXStudio" editor
% !TeX encoding = UTF-8
% !TeX TS-program = lualatex
% !TeX TXS-program:bibliography = biber
% Document title: ETSETB TFG LaTeX Template
% Version: 5.1
% Author: 2023 Orestes Mas Casals
% License: ETSETB TFG LaTeX Template by Orestes Mas is marked with CC0 1.0 Universal

\documentclass[a4paper,twoside,12pt]{report}

% Load auxiliary packages for various tasks
\input{config/packages.tex}

% Document style definitions. GHANGE STYLES THERE.
\input{config/styles.tex}

%%%%%%%%%%%%%%%%%%%%%%%%%%%%%%%%%%%%%%%%%%%%%%%%%%%%%%%%%%%%%%%%%%%%%%
%%%                                                                %%%
%%%              DADES DEL DOCUMENT / DOCUMENT DATA                %%%
%%%                                                                %%%
%%%%%%%%%%%%%%%%%%%%%%%%%%%%%%%%%%%%%%%%%%%%%%%%%%%%%%%%%%%%%%%%%%%%%%

%%% Please UNCOMMENT and/or provide suitable information to the macros below
%%% Si us plau, DESCOMENTEU i/o ompliu les macros següents amb la informació adient.

%%% Specify the document language
%%%   1: Catalan
%%%   2: Spanish
%%%   3: English
\newcommand{\doclanguage}{3}

% Set-up language and load language-related cover page macros
\input{config/lang.tex}

%%% Thesis Title / Títol del TFG
%\setTitle{The title of the thesis, in the desired language}

%%% Name of the Thesis Author / Nom de l'autor del TFG
%\setAuthor{John Doe}

%%% Degree's name. Uncomment the "setDegree" macro and set its argument to one:
%%% Nom del Grau. Descomenteu la macro «setDegree» i passeu-li un d'aquests paràmetres:
%%%   \GEF     - Physics Engineering            <<<< DON'T USE THIS, as the template is not approved by the GEF faculty yet
%%%   \GREELEC - Electronics Engineering
%%%   \GRETST  - Telecommunications Engineering
%\setDegree{\GRETST}

%%% Name of the Thesis Advisor / Director
%\setAdvisor{Joe Q. Public}

%%% Name of the Thesis Rewiewer / «Ponent»
%\setReviewer{Mary Major}

%%% Document's date (\TFGdate by default)
%\setDate{\TFGdate}


%%%%%%%%%%%%%%%%%%%%%%%%%%%%%%%%%%%%%%%%%%%%%%%%%%%%%%%%%%%%%%%%%%%%%%%%%%
%%%%%%%%%%%%%%%%%%%%% PREAMBLE ENDS - DOCUMENT BEGINS %%%%%%%%%%%%%%%%%%%%
%%%%%%%%%%%%%%%%%%%%%%%%%%%%%%%%%%%%%%%%%%%%%%%%%%%%%%%%%%%%%%%%%%%%%%%%%%

\begin{document}

%%%%%%%%%%%%%%%%%%%%%%%%% FRONT COVER %%%%%%%%%%%%%%%%%%%%%%%%%
\begin{titlepage}
  % Cover page style defined in config/styles.tex
  \loadgeometry{cover}
  \thispagestyle{frontcover}
  % Use a Sans Serif typeface throughout
  \sffamily
  
  % Title page background image
  \begin{tikzpicture}[remember picture,overlay]
    \node[anchor=center,opacity=0.3,xshift=5mm,yshift=-10mm,scale=8] (Campillo) at (current page.center) {\includegraphics{img/Campillo.png}};
    \filldraw[color=DegreeColor] (current page.south west) rectangle +(1.5cm,\paperheight);
    \filldraw[color=ETSETBcopper] (current page.south west) -- ++(0,0.140\paperheight-10pt) -- ++(0.75cm,20pt) --   ++(0.75cm,-20pt) -- ++(0,-0.140\paperheight+10pt) -- cycle;
    \node[anchor=south,at=(current page.south west),xshift=0.75cm,yshift=0.5cm] {\rotatebox{90}{\textcolor{white}{\Huge\bfseries \raisebox{0pt}{ETSETB} -- \LARGE\DegreeName}}};
  \end{tikzpicture}
  
  \begin{center}
    \vspace*{2.5cm}
    
    {\Huge \ThesisTitle}\\
    
    \vspace{0.8cm}
    
    \color{black}\hrule height 2pt
    
    \vspace{1cm}
    
    {\large \FirstParagraph}\bigskip\\
  
    {\Large \AuthorName}\medskip\\
    
    \vspace{2.5cm}
    
    \SecondParagraph\medskip\\
    
    {\bfseries\scshape\DegreeName}
    
    % This is an elastic space
    \vfill
    
    \AdvisorLine\medskip\\
    
    \vspace{1cm}
    
    {Linköping, \documentDate}
    
    \vspace{1cm}
  \end{center}
\end{titlepage}

% Restore default page geometry (modified by cover page code)
%\restoregeometry
\loadgeometry{main}

%%%%%%%%%%%%%%%%%%%%%%%%% PREAMBLE BEGINS %%%%%%%%%%%%%%%%%%%%%%%%%
\pagestyle{plain}


%%%% SUMMARY / RESUM %%%%
% DON'T CHANGE THIS LINE
\addcontentsline{toc}{section}{\ifcase\doclanguage\or Resum \or Resumen \else Summary\fi}

%%%% PLEASE REPLACE TEXTS WITH YOUR OWN CONTENT %%%%

%%% RESUM EN CATALÀ
\begin{center}
  \huge\bfseries\raggedright Resum~\hrulefill
\end{center}
  %Cada exemplar del Treball de Fi de Grau (TFG) ha de contenir un Resum, que és un breu extracte del TFG. En termes d'estil, el Resum hauria de ser una versió reduïda del projecte: una introducció concisa, un compendi dels resultats i les principals conclusions o arguments presentats en el projecte. El Resum no ha de superar les 150 paraules i cal que estigui traduït al català, castellà i anglès.
  
  Aquest projecte té com a objectiu millorar l'eficiència i la usabilitat de grans models d'IA, específicament els models d'aprenentatge profund utilitzats en Visió per Computador quan es despleguen en dispositius perimetrals amb recursos limitats. El nostre objectiu és reduir les demandes computacionals d'aquests models sense sacrificar el rendiment desenvolupant una cadena modular que integra les tècniques de destil·lació de coneixement, l'ajustament de paràmetres amb LoRA i quantificació. El projecte quantifica els compromisos entre el rendiment i l'eficiència del model en cada etapa de la cadena. Això s'aconsegueix integrant tres metodologies bàsiques per reduir un gran model d'IA (ResNet-18) a una versió compacta i eficient (MobileNetV2).

%%% RESUMEN EN CASTELLANO
\begin{center}
  \huge\bfseries\raggedleft\vspace*{.5\baselineskip} \hrulefill ~Resumen
\end{center}
  Este proyecto busca mejorar la eficiencia y la usabilidad de grandes modelos de IA, en concreto los modelos de aprendizaje profundo utilizados en Visión Artificial al implementarse en dispositivos perimetrales con recursos limitados. Buscamos reducir la demanda computacional de estos modelos sin sacrificar el rendimiento mediante el desarrollo de una canalización modular que integra técnicas de destilación de conocimiento, ajuste de parámetros con LoRA y cuantificación. El proyecto cuantifica el intercambio entre el rendimiento y la eficiencia del modelo en cada etapa de la canalización. Esto se logra integrando tres metodologías principales para reducir un gran modelo de IA (ResNet-18) a una versión compacta y eficiente (MobileNetV2).

%%% ENGLISH SUMMARY
\begin{center}
  \huge\bfseries\raggedright\vspace*{.5\baselineskip} Summary~\hrulefill
\end{center}
  This project aims to enhance the efficiency and usability of large AI models, specifically deep learning models used in Computer Vision when deployed on edge devices with limited resources. We aim to reduce the computational demands of these models without sacrificing performance by developing a modular pipeline that integrates knowledge distillation, LoRA fine-tuning, and quantization techniques. The project quantifies the trade-offs between model performance and efficiency at each stage of the pipeline. This is achieved by integrating three core methodologies to shrink a large AI model (ResNet-18) into a compact, efficient version (MobileNetV2).
                   % <<<<<< Please replace "summary.tex" entirely with your own content written in the required languages.


%%%% DEDICATION PAGE / PÀGINA DE DEDICATÒRIA %%%%
\input{dedication}


%%%% ACKNOWLEDGEMENTS / AGRAÏMENTS %%%%
\clearpage
\ifcase\doclanguage    % REPLACE FROM HERE (including the \ifcase structure) WITH YOUR OWN TEXT
\or
\chapter*{Agraïments}
\addcontentsline{toc}{section}{Agraïments}

És apropiat, però no obligatori, declarar l’extensió de l’ajuda aportada per persones de l'\textit{staff}, companys/companyes d’estudis, tècnics/ques o altres en la co\l.lecció de dades, disseny i construcció del prototip, l’anàlisi de dades, l’execució dels experiments i la preparació del projecte (incloent l’ajuda editorial). A més a més, és apropiat reconèixer la supervisió i la direcció donada pel tutor/a.
\else
%\chapter*{Acknowledgements}
%\addcontentsline{toc}{section}{Acknowledgements}

%It is appropriate, but not mandatory, to declare the extent to which assistance has been given by members of the staff, fellow students, technicians or others in the collection of materials and data, the design and construction of apparatus, the performance of experiments, the analysis of data, and the preparation of the thesis (including editorial help). In addition, it is appropriate to recognize the supervision and advice given by your advisor.
\fi



%%%% REVISION HISTORY TABLE %%%% <<<<<< Please read the instructions inside the "revision_history.tex" file.
\input{revision_history}


%%%% INDEX %%%%
\clearpage
\addcontentsline{toc}{section}{\ifcase\doclanguage\or Índex\or Índice\else Contents\fi}
\tableofcontents


%%%% LISTS %%%%
\listoffigures 
\addcontentsline{toc}{section}{\ifcase\doclanguage\or Índex de figures \or Lista de figuras \else List of figures\fi}
\listoftables
\addcontentsline{toc}{section}{\ifcase\doclanguage\or Índex de taules \or Índice de cuadros \else List of tables\fi}
%\lstlistoflistings


%%%% ACRONYMS %%%% <<<<<< PLEASE DEFINE YOUR ACRONYMS IN "acronyms.tex"
\clearpage
\newcommand\AbbrvName{\ifcase\doclanguage\or Sigles i acrònims\or Siglas y acrónimos\else Abbreviations\fi}
\addcontentsline{toc}{section}{\AbbrvName}
\printacronyms[
  heading=chapter*,
  display=all,
  include=abbrev,
  name=\AbbrvName,
]


%%%%%%%%%%%%%%%%%%%%%%%%%%%%%%%%%%%%%%%%%%%%%%%%%%%%%%%%%%%%%%%%%%%%
%%%%%%%%%%%%%%%%%%%%%%%%% MAIN PART BEGINS %%%%%%%%%%%%%%%%%%%%%%%%%
%%%%%%%%%%%%%%%%%%%%%%%%%%%%%%%%%%%%%%%%%%%%%%%%%%%%%%%%%%%%%%%%%%%%

% Starts normal numbering for the rest of the document
%\mainmatter  (valid/useful only if using the "book" class instead of "report" in the very first line)
\pagestyle{main}


% Introduction. Please replace "introduction.tex" entirely with your own content written in the desired language.
%%%% PLEASE REPLACE ENTIRELY WITH YOUR OWN CONTENT %%%%

\ifcase\doclanguage
\or
  \chapter{Introducció}
\else
  \chapter{Introduction}
\fi

\ifcase\doclanguage
\or
\section{Objectius del treball}
\else
\section{Motivation}
\fi

This project aims to enhance the efficiency and usability of large AI models, specifically deep learning models used in Computer Vision (CV), when deployed on edge devices with limited resources. The aim is to reduce the computational demands of these models without sacrificing performance by developing a modular pipeline that integrates knowledge distillation, LoRA fine-tuning, and quantization techniques. The project systematically quantifies the trade-offs between model performance (accuracy) and efficiency (size, latency, computational cost) at each stage of the pipeline, applying information-theoretic principles for efficient knowledge transfer and exploring its applicability within a federated edge learning context.

This will be achieved by integrating three core methodologies to shrink a large AI model (ResNet-18) into a compact, efficient version (MobileNetV2):

\begin{enumerate}
	\item Knowledge Distillation: Train a small AI model to mimic a larger, more complex one by transferring the teacher’s knowledge into a student. 
	\item LoRA Fine-Tuning: Adjust the small model for specific tasks without increasing unnecessary size. 
	\item Quantization: Simplify how the model handles numbers, cutting memory usage and battery drain.
\end{enumerate}

The project empirically validates this pipeline, providing a clear methodology for developing lightweight, task-specialized AI suitable for edge computing applications. The methodology is grounded in information theory to maximize knowledge transfer and is designed with federated edge learning in mind, enabling collaborative training without centralizing user data.

\ifcase\doclanguage
\or
  \section{Objectius del treball}
\else
  \section{Work goals}
\fi

The purpose of this project is to address the deployment of large, powerful AI models on resource-constrained edge devices (e.g., smartphones and IoT sensors) by designing, implementing, and evaluating a modular, three-stage pipeline. This pipeline systematically combines knowledge distillation, Low-Rank Adaptation (LoRA), and quantization to create a lightweight yet functional version of a standard computer vision model.

Furthermore, the project will be grounded in information theory to guide the distillation process, ensuring that the student model learns to compress the input while preserving the maximum information about the teacher's predictions. %The pipeline is also designed for future deployment in a Federated Edge Learning environment. In this setting, 
Knowledge distillation is a mechanism that enables communication-efficient and privacy-preserving collaborative training across multiple devices.

To achieve the main goal of developing and evaluating a compressed student model, the project has the following specific objectives:

\renewcommand{\labelenumii}{\arabic{enumi}.\arabic{enumii}}
\renewcommand{\labelenumiii}{\arabic{enumi}.\arabic{enumii}.\arabic{enumiii}}
\renewcommand{\labelenumiv}{\arabic{enumi}.\arabic{enumii}.\arabic{enumiii}.\arabic{enumiv}}
\begin{enumerate}
	\item Hybrid Distillation-LoRA Framework: Develop a two-phase approach where a compact student model is first distilled from a large pre-trained AI model and then fine-tuned using low-rank adaptation to achieve task-specific performance improvements.
	\begin{enumerate}
		\item Develop a mini-model for hardware with limited resources: Create a compact model based on a method to distill knowledge from a large AI into a smaller, lighter model that can run efficiently on T4 Colab.
		\item Optimize this Lightweight Version to minimize memory usage for CV by Fine-Tuning with Low-Rank Adaptation: Use techniques like LoRA to further optimize the smaller model for specific tasks, improving its performance without increasing its size significantly.
	\end{enumerate}
	\item Reduce Computational Costs through Quantization and Edge Optimization: Apply quantization strategies to shrink the model’s size and lower its energy consumption, making it suitable for real-world applications. In other words, implement quantization-aware distillation methods to reduce model size and computational complexity, ensuring viability for edge deployment. 
	\item Evaluate the resulting AI model against the original using standard metrics for speed, memory usage, and accuracy:
	\begin{enumerate}
		\item Evaluate Performance: Test the optimized model on practical tasks (e.g., object classification), analyzing how well it balances size, speed, and accuracy.
		\item Study Fine-Tuning Dynamics: Investigate how the initial distillation phase impacts the efficiency and effectiveness of subsequent LoRA-based fine-tuning, identifying the most critical elements of the original large AI model knowledge for downstream performance.
	\end{enumerate}
	The evaluation will be done in parallel with knowledge distillation and fine-tuning. 
	\item Frame the distillation process within an information-theoretic context: Analyze knowledge transfer within an information-theoretic context, and explore extending the above pipeline to a Federated Learning setting for decentralized, privacy-preserving training.
\end{enumerate}

\ifcase\doclanguage
\or
  \section{Requisits i especificacions}
\else
  \section{Requirements and specifications}
\fi

A set of requirements has been defined to ensure that the developed system meets the objectives, covering key points such as the model’s capabilities and its evaluation.

\textbf{Requirements}

\begin{table}[H]
	\resizebox{\columnwidth}{!}{
		\begin{tabular}{|m{3.5cm}|p{11cm}|}\hline
			\rowcolor{maroon}
			\textbf{Requirement} & \textbf{Description} \\ \hline
			Benchmark & The distilled model must retain high accuracy compared to their teacher while being deployable on resource-constrained edge devices. \\ \hline
			Efficiency & The lightweight model must demonstrate reduced computational cost and inference time. \\ \hline
			Pipeline & The pipeline must integrate knowledge distillation, LoRA fine-tuning and quantization. It must be reproducible and modular for future extensions. \\ \hline
			Evaluation & The resulting model must be evaluated on standard CV benchmarks and framed within an information-theoretic perspective. \\ \hline
		\end{tabular}
	}
	\caption{Requirements table}
	\label{table:requirements_table}
\end{table}

\textbf{Specifications}

\begin{table}[H]
	\resizebox{\columnwidth}{!}{
		\begin{tabular}{|p{3.5cm}|p{11cm}|}\hline
			\rowcolor{maroon}
			\textbf{Specification} & \textbf{Description} \\ \hline
			Accuracy & The student model must achieve 90\% accuracy of the teacher model.                                                                                     \\ \hline
			Model size & The model must reduce \textgreater{}75\% of the original size.                                                                                  \\ \hline
			Time optimization     & The model must reduce inference time. \\ \hline
			Development platform         & The developed system must be compatible with major operating systems such as Linux, macOS, Windows. \\ \hline
		\end{tabular}
	}
	\caption{Specifications table}
	\label{table:specifications_table}
\end{table}

\textbf{Model Architectures:}

The selection of teacher and student models is designed to create a realistic and challenging distillation scenario.

\begin{list}{-}{}
	\item \textbf{Teacher Model: ResNet-18:} This architecture is selected for its known high performance in CV, since it can learn complex, non-linear channel features, establishing a strong upper-bound for accuracy.
	\item \textbf{Student Model: MobileNetV2:} This model is chosen as a prototypical lightweight architecture for edge devices. Its efficiency comes from the use of depthwise separable convolutions, which drastically reduce the parameter count and computational load compared to standard convolutions. The significant architectural differences between ResNet-18 and MobileNetV2 make them excellent test cases for evaluating the effectiveness of different distillation techniques.
\end{list}
 
\ifcase\doclanguage
\or
  \section{Mètodes i procediments}
\else
  \section{Methods and procedures}
\fi

The project builds upon existing state-of-the-art technologies, architectures and methods developed by other authors:

\begin{list}{-}{}
	\item Teacher model: ResNet-18 (He et al., 2015)
	\item Student model: MobileNetV2 (Sandler et al., 2018)
	\item Distillation methods: based on Hinton et al. (2015)
	\item LoRA fine-tuning: Low-rank adaptation methods (Hu et al., 2021)
\end{list}

The project is conducted independently and does not form part of any department or company research or development project. The UPC supervisor provided the foundational concepts and initial ideas, which were then complemented by the supervisor at Linköping University.

\ifcase\doclanguage
\or
  \section{Pla de treball}
\else
  \section{Work plan}
\fi
\label{sec:workplan}

\ifcase\doclanguage
\or
  Normalment les figures i taules es col·loquen en els entorns \verb|\figure| i \verb|\table|, que poden flotar lliurement en el document. Pots identificar cada flotant amb un \verb|\label|
\else

  The project is structured into three overlapping phases over 17 weeks (September 1, 2025 – December 20, 2025), ensuring a clear progression from foundational development to advanced analysis.
  
  Distillation-LoRA Framework \& Initial Evaluation. This phase focuses on implementing the core compression pipeline. The key objective is a functional, fine-tuned student model benchmarked against the original teacher and baseline student.
  
  Quantization \& Continued Evaluation. This phase builds on Phase 1 by applying the final compression stage, quantization. The objective is the final, fully compressed model, and a complete set of experimental results.
  
  Theoretical Analysis, Final Report, and Submission. This final phase is dedicated to the theoretical framing of the project, in-depth analysis of the results, and writing the thesis document.
   
  Work Packages
  
  \begin{itemize}
  	\item WP1: Distillation-LoRA Framework \& Evaluation
  	\begin{itemize}
  		\item Tasks:
  		\begin{itemize}
  			\item T1.1: Comprehensive literature review on KD, LoRA, and quantization.
  			\item T1.2: Setup of development environment (PyTorch, Colab).
  			\item T1.3: Implementation of data loading and preprocessing for CIFAR-100.
  			\item T1.4: Train and benchmark the ResNet-18 teacher and baseline MobileNetV2 student.
  			\item T1.5: Implement the knowledge distillation framework and train the distilled student model.
  			\item T1.6: Implement LoRA fine-tuning on the distilled student model.
  			\item T1.7: Benchmark all intermediate models (teacher, baseline, distilled, LoRA-tuned).
  		\end{itemize}
  	\end{itemize}
  	\item WP2: Quantization \& Theoretical Framing
  	\begin{itemize}
  		\item Tasks:
  		\begin{itemize}
  			\item T2.1: Apply post-training INT8 quantization to the LoRA-tuned model.
  			\item T2.2: Design and execute a final, comprehensive benchmarking script to evaluate all model versions on accuracy, size, latency, and FLOPs.
  			\item T2.3: Generate all tables and visualizations for the results chapter.
  			%\item T2.4: Analyze knowledge transfer from an information-theoretic perspective.
  			%\item T2.5: Define and describe how the pipeline can be extended for Federated Edge Learning.
  		\end{itemize}
  	\end{itemize}
  	\item WP3: Project Management \& Thesis Writing
  	\begin{itemize}
  		\item Tasks:
  		\begin{itemize}
  			\item T3.1: Draft Introduction and State of the Art chapters.
  			\item T3.2: Draft Methodology chapter as WP1 and WP2 tasks are completed.
  			\item T3.3: Draft Results, Sustainability Analysis, and Conclusion chapters.
  			\item T3.4: Final review, formatting, and thesis submission.
  		\end{itemize}
  		\item Primary Deliverables:
  		\begin{itemize}
  			\item Proposal Presentation and TFG Work Plan. deadline: 05/10/2025
  			\item Critical Review. deadline: 30/11/2025
  			\item Final Bachelor's Thesis document. deadline: 18/01/2026
  		\end{itemize}
  	\end{itemize} 	
  \end{itemize}
  %Below is the Gantt diagram with the timeline of the project.
  %Normally the figures and tables are put in \verb|\figure| and \verb|\table| environments, that can float freely in the document. You can identify each float with a \verb|\label|
\fi

\begin{figure}[ht]
  %\centering
  %\includegraphics[width=1\textwidth]{../../Pictures/Screenshots/gantt}
%  \input{img/gantt_diagrama}
%  \ifcase\doclanguage
%  \or
%    \caption[Diagrama de Gantt del projecte]{\footnotesize{Diagrama de Gantt del projecte. Per a més informació, llegiu el manual \cite{skalagantt} de Skala.}}
%  \else
%    \caption[Project's Gantt diagram]{\footnotesize{Gantt diagram of the project}}
%  \fi
%  \label{fig:gantt}
\end{figure}




% State of the art chapter. Please replace "state_of_art.tex" entirely with your own content written in the desired language.
%%%% PLEASE REPLACE ENTIRELY WITH YOUR OWN CONTENT %%%%

%\ifcase\doclanguage
%  \chapter[Estat de l'art]{Estat de l'art de la tecnologia utilitzada o aplicada en aquest TFG}

%  El capítol «Estat de l'Art de la Tecnologia» ofereix una visió detallada dels avenços actuals relacionats amb el tema del vostre treball. Hauria de descriure les teories, models, algorismes clau o desenvolupaments de programari i maquinari, recolzats per articles revisats per experts i altres recursos com ara llibres, patents, informes tècnics, etcètera. Aquest capítol estableix el context i ajuda els lectors a entendre el panorama existent en el camp. En conseqüència, les \textbf{cites a referències bibliogràfiques rellevants} són una part important del contingut.
  
%  Aquest capítol no només ha de resumir la recerca existent, sinó que també ha de fer-ne una \textbf{avaluació crítica}. Això implica destacar les llacunes o limitacions en la tecnologia actual per preparar el terreny per a les tasques que el TFG abordarà. Al final del capítol els lectors haurien de tenir una comprensió clara del que ja se sap sobre el tema, del que encara queda per aprendre i de com el TFG contribueix a aquesta «conversa acadèmica» en curs.

%  \section{Apartat 1}

%  Aquí teniu un parell de cites a referències sobre \LaTeX~\cite{latexcompanion} i electrodinàmica \cite{einstein}.

%\else
  \chapter[State of the art]{State of the art of the technology applied in this thesis}

  % The "Technology State of the Art" chapter offers a detailed view of the current advancements related to the topic of your work. It should describe key theories, models, algorithms, or developments in software and hardware, supported by peer-reviewed articles and other resources such as books, patents, technical reports, etc. This chapter sets the context and helps readers understand the existing landscape in the field. Consequently, the \textbf{citations to relevant bibliographic references} are an important part of the content.
  
  % This chapter should not only summarize existing research but also provide a critical evaluation of it. This involves highlighting the gaps or limitations in current technology to set the stage for the tasks that the bachelor's thesis will address. By the end of the chapter, readers should have a clear understanding of what is already known about the topic, what still needs to be learned, and how the bachelor's thesis contributes to this ongoing "academic conversation."

  In order to achieve better performance, current deep learning models generally are deeper and wider. However, these heavy models are hard to deploy on resource-constrained devices in practice due to computational and memory resource limitations. We aim to develop an AI model that is smaller, faster, and more energy-efficient while maintaining its intelligence. In the following sections, we follow the development of deep learning, making a survey and integrating three core methodologies to shrink a large AI model (ResNet-18) \cite{he2015deepresiduallearningimage} into a compact, efficient version (MobileNetV2) \cite{sandler2019mobilenetv2invertedresidualslinear}: Knowledge Distillation \cite{hinton2015distillingknowledgeneuralnetwork}, LoRA Fine-Tuning \cite{hu2021loralowrankadaptationlarge} and Quantization.
  
  This project empirically validates this pipeline, providing a clear methodology for developing lightweight, task-specialized AI suitable for edge computing applications. The methodology is grounded in information theory to maximize knowledge transfer and is designed with Federated Edge Learning (FEL) \cite{wu2024knowledgedistillationfederatededge} in mind, enabling collaborative training without centralizing user data.

  \section{Deep Learning}

  Deep learning is a subset of machine learning that utilizes multilayered neural networks to simulate decision-making of the human brain to perform tasks such as classification, regression and feature learning. Deep neural networks consist of multiple layers of interconnected nodes, a combination of data inputs, weights and bias, which can be expressed as $W^T·X+b$.
  
  Over the past decade, deep learning has revolutionized computer vision. Architectures such as ResNet have set new records in benchmark performance. However, these deep learning models are computationally intensive and require large amounts of memory and processing power, which limits their deployment to high-performance servers with dedicated GPUs or TPUs.
  
  On the other hand, edge devices such as smartphones, IoT sensors, drones, and medical wearables, have resource constraints, including limited processing power, memory, and energy. Despite these constraints, edge computing is becoming increasingly important because it reduces latency, lowers dependence on centralized cloud services, and improves privacy by keeping sensitive data local. A key challenge in this field is finding a way to leverage the power of deep learning models while addressing the limitations of edge devices.

  \section{Model compression}
  
  The deployment of advanced AI models on edge devices with limited resources has motivated substantial research into model compression techniques. The central goal of this field is to balance the power of large models with the requirements of efficiency, latency and privacy. This field has evolved from applying individual optimization techniques in isolation to designing integrated, multi-stage pipelines that synergistically combine methods for maximum efficiency.
  
  There are several ways to make deep learning models more efficient without sacrificing accuracy with the following core components of this project: Knowledge Distillation (KD) shrinks the knowledge base \cite{hinton2015distillingknowledgeneuralnetwork}, Parameter-Efficient Fine-Tuning (PEFT) \cite{zhang2025parameterefficientfinetuningfoundationmodels} with a specific focus on Low-Rank Adaptation (LoRA) fine-tunes for specific tasks \cite{hu2021loralowrankadaptationlarge}, and Quantization further compresses the model for deployment.
  
  In this integration, we combine multiple AI model compression techniques in sequence for developing methodologies; instead of applying these methods in isolation, we view them as interconnected steps in a single, integrated process, in which the order and interaction of these techniques are considered crucial for maximizing efficiency and achieving the best compression results, as exemplified by approaches that combine distillation, LoRA, and pruning in a multi-stage process.
  
  Moslem's recent work on speech translation combines knowledge distillation with QLoRA and iterative pruning in a multi-stage process \cite{moslem-2025-efficient}. This highlights the importance of technique order, as distillation can create a more robust student model, aiding subsequent quantization or fine-tuning.
  
  \section{Knowledge Distillation}
  
  A critical analysis of seminal and recent works reveals a clear trajectory in the field. Initially, techniques were developed to solve singular problems, e.g. KD for knowledge transfer and LoRA for tuning efficiency.
  
  Hinton, Vinyals, \& Dean focused exclusively on the concept of transferring "dark knowledge" from a teacher to a student model, establishing the theoretical groundwork for response-based distillation \cite{hinton2015distillingknowledgeneuralnetwork}. Although their work showed that a student model could learn a richer structure by training on the teacher's outputs, they didn't address the efficiency of adapting the distilled model to new tasks.
  
  Knowledge Distillation is a machine learning technique that aims to transfer the learnings of a large pre-trained model to a smaller and more compact one so that the “student” mimics the behaviour of the “teacher” \cite{hinton2015distillingknowledgeneuralnetwork} by matching its predictions. It is usually applied to deep neural networks with many layers and learnable parameters. The main idea is to use soft probabilities of the larger network to supervise the smaller one, which reveal more information than the class labels alone. Soft targets can be estimated by a softmax function as:
  
  % Softened softmax with temperature
  \[
  q_i = 
  \frac{\exp\left(z_i / T\right)}
  {\sum_j \exp\left(z_j / T\right)}
  \]
  
  The loss for the student is then a linear combination of cross entropy loss $L_{CE}$ and knowledge distillation loss $L_{KD}$.
  % Combined hard-label + distillation loss
  \[
  L = 
  (1 - \alpha)\,L_{\text{CE}}(y, p_{\text{student}})
  + \alpha\,
  L_\mathrm{KD}\!\left(
  p_{\text{teacher}}, p_{\text{student}}
  \right)
  \]
  
  Knowledge Distillation techniques have been employed across multiple fields, including speech recognition, natural language processing, image recognition \cite{he2015deepresiduallearningimage} and object detection. KD is an effective means of transferring the capabilities from a large model or set of models to a single smaller model.
  
  AI model's accuracy and capacity are not enough to make the model useful - it has a limit of time, memory and computational resources. While top performing models are often too large for most applications, small models are faster yet lack the accuracy and knowledge capacity of the first.
  
  \section{Low-Rank Adaptation}
  
  Low-Rank Adaptation (LoRA) was originally developed for efficient fine-tuning in large language models (LLMs). LoRA offers an opportunity to fine-tune compact models without enlarging their size. Hu et al. developed LoRA to address the computational infeasibility of fine-tuning large models, reducing trainable parameters, memory usage and training time by freezing pre-trained weights and training only small, low-rank adapter matrices inserted into existing weight structures, without increasing inference latency \cite{hu2021loralowrankadaptationlarge}. The pre-trained model’s original weight matrix is frozen and only the smaller matrices are updated during training. Yet, the LoRA paper did not focus on reducing the size of the base model itself, which remained a significant barrier for edge deployment.
  
  Instead of retraining the whole model, LoRA freezes the original weights and parameters of the model. On top of this original model, it adds a lightweight addition called a low-rank matrix, which is then applied to new inputs to get results specific to the context. The low-rank matrix adjusts for the weights of the original model so that outputs match the desired use case.
  
  The diagram shows how the additional matrices A and B are updated by using smaller matrices of rank r. Once LoRA training is complete, smaller weights are merged into a new weight matrix, without modifying the original weights W of the pre-trained model and not needing to store a complete copy of the model. In the training process after calculating the loss, the loss is only backpropagated to the LoRA matrices.
  [insert diagram of LoRA]

  Full fine-tuning requires storing optimizer states for all parameters. Since LoRA freezes most of the model's weights, GPU memory usage is lower than full fine-tuning
  storage efficiency training speed.

  Forward pass:
  \[ h = W_0 x + \Delta W x = W_0 x + BAx \]
  
  \section{Quantization}
  
  The limitations of these isolated approaches naturally led to hybrid solutions such as QLoRA \cite{dettmers2023qloraefficientfinetuningquantized}, which combines 4-bit quantization and LoRA adapters, enabling the fine-tuning of large models on a single consumer GPU. This unified approach addresses memory issues in both base models and fine-tuning, shifting towards combined compression solutions.
  
  Quantization reduces the precision of model weights and activations (e.g., from 32-bit floating-point to 8-bit integers), which reduces memory and speeds up inference time. Q
  
  Post-training quantization (PTQ) applies quantization to a fully trained model without additional training. It typically relies on a small calibration dataset to estimate activation ranges and is widely used due to its simplicity and low computational cost. While PTQ can introduce accuracy degradation it is often sufficient for inference-only deployments where retraining is impractical.
  
  In contrast, quantization-aware training (QAT) applies quantization during training, allowing the model to adapt to reduced precision and mitigating accuracy degradation. This approach generally achieves higher accuracy than PTQ but requires access to training data and increased training complexity.
  
  \section{Federated Edge Learning}
  
  In many real-world applications (e.g., healthcare, smart cities), it is impractical to centralize user data for privacy concerns. Federated learning (FL) enables decentralized training across distributed devices while keeping raw data local. However, FL aggravates the need for efficient models because each device must independently train and update parameters under limited resources. By combining KD, LoRA, and quantization, the proposed pipeline aligns with federated edge learning, ensuring that lightweight models can still learn collaboratively without overwhelming device capabilities.
  
  The evolution of compression pipelines is further enriched by integrating principles from information theory and paradigms like Federated Edge Learning (FEL) \cite{wu2024knowledgedistillationfederatededge}. Knowledge distillation, seen through an information-theoretic perspective, aims for a student model to compress input while maximizing information from the teacher's prediction, which implies optimal teachers might be intermediate models, preserving "dark knowledge" — mutual information crucial for generalization often lost in fully converged models \cite{wang2022efficientknowledgedistillationmodel}.
  
  Concurrently, the rise of FEL has positioned knowledge distillation as a critical enabler for privacy-preserving, decentralized machine learning. In Federated Distillation (FD), instead of exchanging high-dimensional model parameters, clients exchange compact model outputs (logits) on a shared, public dataset, allowing heterogeneous models to collaboratively train without sharing private data \cite{wu2024knowledgedistillationfederatededge}.
  
  This project meets at the intersection of these trends, using information-theoretic concepts to guide the distillation process and structuring the pipeline in a way that is directly applicable to a future federated deployment.
  
  While these multi-stage pipelines are emerging, particularly in natural language processing, a systematic, empirical study of a sequential KD → LoRA → Quantization pipeline for computer vision remains a clear research gap.






% Methodology chapter. Please replace "methodology.tex" entirely with your own content written in the desired language.
%%%% PLEASE REPLACE ENTIRELY WITH YOUR OWN CONTENT %%%%

\ifcase\doclanguage
\or
  \chapter{Metodologia / desenvolupament del projecte}
  
  En aquest capítol es detallarà la metodologia emprada en la realització del treball. Té com a objectiu oferir un compte detallat de les aproximacions i tècniques utilitzades, assegurant la replicabilitat i el rigor acadèmic. No només cobrirà els mètodes de recerca i tècniques de mesurament emprats, sinó que també aprofundirà en les especificitats del desenvolupament de programari i maquinari. Tant si el projecte implica anàlisi qualitativa, mesuraments quantitatius, modelatge computacional com prototipatge físic, aquest capítol hauria d'elucidar com contribueix cada component als objectius generals.
  
  A més de descriure els mètodes en si mateixos, el capítol també proporcionarà justificacions per què es van escollir mètodes particulars enfront d'altres. Per exemple, podria explicar la tria d'un llenguatge de programació específic, prova estadística o configuració experimental. El capítol també abordarà les limitacions de la metodologia i com aquestes s'han mitigat o tingut en compte. Els lectors haurien de sortir amb una comprensió clara de com s'ha dut a terme el desenvolupament del projecte, per què s'han escollit determinades opcions i com aquests mètodes serveixen per complir els objectius establerts inicialment.
  
\else
  \chapter{Methodology / Project Development}

  %In this chapter, the methodology used in the completion of the work will be detailed. Its aim is to offer a thorough account of the approaches and techniques used, ensuring replicability and academic rigor. It will not only cover the research methods and measurement techniques employed but will also delve into the specifics of software and hardware development. Whether the project involves qualitative analysis, quantitative measurements, computational modeling, or physical prototyping, this chapter should elucidate how each component contributes to the overall objectives.
  
  %In addition to describing the methods themselves, the chapter will also provide justifications for why specific methods were chosen over others. For example, it may explain the choice of a particular programming language, statistical test, or experimental setup. The chapter will also address the limitations of the methodology and how these have been mitigated or accounted for. Readers should come away with a clear understanding of how the project's development has been carried out, why certain choices were made, and how these methods serve to fulfill the initially established objectives.
  
  This chapter describes the methodology employed to design, train and evaluate the knowledge distillation framework proposed in this study. It outlines the dataset, data preparation, teacher and student model architectures, training procedures, evaluation metrics and experimental setup. %Each section provides explanation of the methods, tools and reasoning behind key decisions.
  
  This research follows an experimental design where a teacher model is used to supervise the training of a smaller student model. A series of experiments evaluate the impact of different distillation strategies.

\fi

\ifcase\doclanguage\or
  \section{Introducció d'expressions matemàtiques}
    
  \LaTeX{} és una eina inestimable per a la composició tipogràfica de contingut matemàtic. En aquesta secció mostrem les comandes i entorns \LaTeX{} essencials per a l'escriptura matemàtica. Per a més informació consulteu el capítol 3 de \cite{notsoshort}.
    
  \subsection{Matemàtiques en línia i aïllades}
    
  Per a expressions en línia, utilitzeu \verb|$ ... $| o \verb|\( ... \)|. Escriviu entre \verb|\[ ... \]| les expressions que s'han de mostrar en una línia apart.

\else
  \section{Data Preparation and Preprocessing}
  
  For the initial evaluations of the model, we conducted experiments to evaluate RestNet18 and MobileNetV2 performance using the public available CIFAR-10 and CIFAR-100 datasets. The CIFAR-10 and CIFAR-100 datasets are labeled subsets of the 80 million tiny images dataset. This dataset was chosen because it is an established computer vision dataset used for object recognition and it allows to quickly try different algorithms, ideal for benchmarking and preliminary evaluations.
  
  Data preparation is a crucial step that transforms the input data into the format required by the model, ensuring it can be correctly processed and learned from.
  
  The images are normalized to ensure consistency in the data format. Data normalization prevents large values from dominating the learning process by ensuring numerical features are on a similar scale for optimal model performance. It avoids numerical instability, speeds up convergence in gradient-based algorithms and ensures features contribute equally.
  
%  The dataset is divided in training, validation and test.
%  These splits are NOT defined by the dataset
  
  \section{Model Training and Fine-Tuning}
  
  Training a model involves teaching it to identify patterns within data so that it can make accurate predictions or decisions when presented with new, unseen inputs. The training process is generally divided into two key stages: pre-training and fine-tuning.
  
  \begin{list}{-}{}
  	\item Pre-training refers to training a model on a large, general-purpose dataset, often using unsupervised or self-supervised learning methods. During this stage, the model develops a foundational understanding of the data by learning to extract features and recognize broad patterns, without focusing on any specific task. The resulting pre-trained models serve as a strong starting point for various downstream applications, significantly reducing both time and computational cost compared to training a model entirely from scratch.
  	\item Fine-tuning, on the other hand, builds upon a pre-trained model by adapting it to a specific task or dataset. This is done by further training the model on a smaller, labeled, and task-specific dataset. Through fine-tuning, the model refines its learned representations and becomes specialized in the target application while still benefiting from the general knowledge acquired during pre-training.
  \end{list}
  
  In this work, the ResNet-18 model serves as the teacher network due to its strong performance on image classification tasks. We focus on fine-tuning the ResNet-18 model rather than training it from scratch. This approach is supported by the availability of high-quality pre-trained checkpoints in PyTorch, which have been trained on large-scale datasets. These pre-trained models already capture fundamental visual features and general data representations. However, their learned knowledge remains shallow and tied to the original training data. Fine-tuning enables the model to adapt and develop more task-specific capabilities. In our case, the teacher was pre-trained on ImageNet and fine-tuned on the target CIFAR dataset.
  
  By leveraging the visual representations obtained during pre-training, this method avoids the significant computational expense and environmental impact associated with large-scale model training. Consequently, it aligns with sustainable AI practices. In our implementation, we fine-tune the ResNet-18 model using labeled data to tailor it to the desired task.
  
  In the distillation process, the student model is a MobileNetV2, selected for its compact size and suitability for edge deployment. The student model is trained for 75 epochs using Adam optimizer with learning rate of 0.001. All hyperparameters are tuned based on the validation set. After several runs, the distillation temperature and alpha values chosen are T=4 and $\alpha=0.3$. It is trained under the supervision of the teacher, using both ground truth labels and soft targets learned from the larger model. %AJUSTAR NUM 
  
  Model performance is evaluated using top-1 accuracy, F1-score, and computational metrics such as inference latency.
  
  % During fine-tuning, the pre-trained model’s weights are frozen for the first NUM of updates. This value provides a good balance between stabilizing the learning process and enabling effective fine-tuning. Freezing the weights for too long may lead to overfitting, which ultimately reduces the model’s ability to generalize.
  
  To select the best checkpoint, we monitor validation accuracy and choose the checkpoint that achieves the highest score. This helps ensure that the model generalizes well to unseen data and avoids overfitting to the training set. Although fine-tuning continues for additional steps, this process ensures that training stops at an optimal point.
  
  \textbf{Experimental Setup}
  	  
  Fine-tuning of the MobileNet model is performed using a custom configuration that defines the parameters used in training, optimization, and overall model behavior. This setup ensures the model can adapt to the specific dataset used in our experiments. The configuration remains consistent across all trained models to maintain uniformity.
  
  Fixed random seed values of 42, 518, 1993 are used to ensure reproducibility, enabling consistent results across multiple runs.
  
  To maintain efficiency during training, only the highest-performing checkpoint is retained. Selection of the best checkpoint is based on the validation loss metric, ensuring that the stored model reflects peak performance for future evaluation.
  
  The model is pre-trained on the ImageNet dataset for image classification. Data pre-processing includes normalization to standardize inputs before they are fed into the model. %	Image augmentation is also applied, introducing random transformations to increase robustness to data variability.
  
  For optimization, we use the Adam optimizer, a widely adopted choice due to its adaptability to changing gradients during training. The model begins training with an initial learning rate of 0.001, chosen to provide stable updates without causing divergence.
  
  \section{Quantization}
  
  Model quantization converts high-precision floating-point values into lower-precision formats such as integers, resulting in faster inference, lower energy consumption, and reduced storage requirements. In particular, model parameters (weights and activations) are approximated using INT8 instead of the more common 32-bit floating-point format (FP32). This is especially beneficial in tasks such as image processing and neural network inference, where computational efficiency and memory optimization are crucial.
  
  \section{Benchmark}
  
  Evaluation is a fundamental component of any research project, as it provides a systematic and objective means of measuring the effectiveness and robustness of a proposed system. A well-designed evaluation framework allows researchers to compare models under consistent conditions, quantify performance using standardized metrics, and identify the strengths and limitations of different approaches.
  
  The quality of the evaluation process depends heavily on the datasets used. These datasets must be both trustworthy and sufficiently diverse to reflect real-world scenarios. Reliable benchmark data ensure that the results are reproducible and not influenced by noise or bias, while variation within the dataset allows for a more comprehensive assessment of the model’s ability to generalize beyond the specific examples seen during training.
  
  \section{Evaluation Framework and Testing}
    
  The evaluation framework used in this project is structured to assess the performance and reliability of the MobileNetV2 student model against the teacher. The framework enables consistent comparison of model behavior under different experimental parameters. This structure not only helps verify the model’s robustness but also ensures that its performance can be generalized beyond the specific conditions encountered during training.
  
  To conduct evaluation, the model is tested on a dataset that is isolated from the training process, therefore ensuring an unbiased assessment. The testing procedure examines the model’s ability to correctly classify inputs and handle variations in data distribution.
   
  Following the evaluation, accuracy is computed as the primary performance metric. Accuracy provides a clear measure of the proportion of correct predictions relative to the total number of samples, making it an appropriate and widely accepted indicator for classification tasks.
  
  %Limitations of Accuracy in Imbalanced Datasets
  However, accuracy is often insufficient for evaluating models trained on imbalanced datasets, where one class substantially outweighs the others. In such cases, a model may achieve deceptively high accuracy simply by consistently predicting the majority class, while failing to detect the minority class altogether. This makes accuracy an unreliable indicator of true model performance.
  To address these limitations, additional metrics such as precision, recall, and the F1 score are employed. These metrics provide more informative insights, particularly regarding the model’s ability to recognize minority classes and to manage trade-offs between false positives and false negatives.
  
  Precision measures the proportion of correctly predicted positive instances out of all instances predicted as positive. High precision indicates that the model’s positive predictions are reliable.  
  
  \[Precision = \frac{True Positives}{True Positives + False Positives}\]
  
  Recall, also known as sensitivity or the true positive rate, quantifies the proportion of actual positive instances that the model successfully identifies. Recall is crucial in scenarios where failing to detect positive cases, producing false negatives, is particularly costly.
 
  \[Recall = \frac{True Positives}{True Positives + False Negatives}\]
  
  Precision and recall frequently exhibit a trade-off. Improving one may reduce the other. Therefore, selecting the appropriate balance depends on the specific requirements and risks associated with the task.
  
  The F1 score provides a single metric that balances both precision and recall, making it particularly useful when false positives and false negatives are equally important. It is defined as the harmonic mean of precision and recall:
  
  \[F1 Score = 2·\frac{Precision·Recall}{Precision+Recall}\]
  
  The F1 score ranges from 0 to 1, with higher values indicating stronger overall performance. Achieving an F1 score of 1 would require perfect precision and recall simultaneously, which is a condition rarely met in practice due to the trade-offs between detecting all positive instances and minimizing incorrect positive predictions.

\fi


%%%% RESULTS %%%%
% Results chapter. Please replace "results.tex" entirely with your own content written in the desired language.
%%%% PLEASE REPLACE ENTIRELY WITH YOUR OWN CONTENT %%%%
\ifcase\doclanguage\or
  \chapter{Resultats}
  Aquest capítol ha d'incloure l'anàlisi de les vostres dades i els resultats obtinguts. A més, incloeu-hi taules, figures i citacions pertinents per donar suport als vostres resultats i interpretacions. Aquí teniu una llista suggerida de temes a tractar:
  
    \section{Experiments i proves}
    Descriviu els experiments realitzats per provar el rendiment del vostre projecte. Expliqueu com heu recopilat i processat les dades.
    
    \section{Visualització de les dades}
    Creeu representacions visuals dels resultats (per exemple, gràfics de dispersió, diagrames de barres). Interpreteu les visualitzacions i relacioneu-les amb les preguntes de recerca.
    
    \section{Limitacions}
    Reconeixeu qualsevol limitació en les dades o l'anàlisi. Expliqueu com aquestes limitacions podrien haver afectat els resultats.
  
\else
  \chapter{Results}
  This chapter outlines the experimental setup used to train and evaluate the model. 
  
    \section{Experiments and Tests}
    Describe the experiments conducted to assess the performance of your project. Explain how you collected and processed the data.
    
    \section{Data Visualization}
    Create visual representations of the results (e.g., scatter plots, bar charts). Interpret the visualizations and relate them to the research questions.
    
    \section{Limitations}
    Acknowledge any limitations in the data or analysis. Explain how these limitations may have influenced the results.

\fi

%
%%%% SECTION3 %%%
%\newpage
%%\vspace*{2cm}
%\section{Section 3}
%\label{sec:sec3}
%
%\lipsum[4] \ac{EU} is the European Union. \lipsum[5]
%\lipsum[6] \ac{ETSETB} is Telecos. \lipsum[7]
%
%\subsection{Subsection}
%\label{sec:subsec3.1}
%The book \cite{latexcompanion} \lipsum[15]
%
%\include{codi_tempdistrib}
%
%%%% SECTION4 %%%
%\input{section4}
%
%%%% SECTION5 %%%
%\newpage
%%\vspace*{2cm}
%\section{Section 5}
%\label{sec:sect5}
%\lipsum[4]
%
%\subsection{Overview}
%\label{subsec:sect5Overview}
%\lipsum[10]
%Visite the Knuth repository \cite{knuthwebsite}.
%
%%%% TESTING %%%
%\clearpage
%%\vspace*{2cm}
%\section{Experiments and results}
%\label{sec:tests}
%\lipsum[9]
%



%%%% BUDGET %%%%
% Budget chapter. Please replace "budget.tex" entirely with your own content written in the desired language.
%\input{budget}


%%%% SUSTAINABILITY REPORT %%%%    <<<<<<<<<<<<<<<< NEW !!
% Sustainability chapter. Please replace "sustainability.tex" entirely with your own content written in the desired language.
%%%% PLEASE REPLACE ENTIRELY WITH YOUR OWN CONTENT %%%%

\ifcase\doclanguage\or
\chapter{Anàlisi de sostenibilitat i implicacions ètiques}
Des del curs 2023-24, la normativa de TFG de l'ETSETB demana la inclusió d'un informe de sostenibilitat a la memòria del treball. Aquesta anàlisi consisteix en una valoració dels impactes ambientals, socials i econòmics, i les possibles implicacions ètiques que ha comportat la realització del TFG. En el cas que el TFG plantegi un producte/servei/sistema/edifici/etc., que podria arribar a implementar-se, l’anàlisi també ha de realitzar-se sobre els impactes que tindria la proposta en la seva execució durant les diferents etapes del seu cicle de vida.

A la plataforma ATENEA trobareu un document separat amb les instruccions detallades de què ha de contenir i com cal confeccionar l'informe de sostenibilitat.

{\bigskip\bfseries {\large IMPORTANT:} Noteu que l'antic capítol de «Pressupost del projecte» ara queda integrat en l'anàlisi de sostenibilitat, concretament en les cel·les «Econòmic/Desenvolupament del TFG» i «Econòmic/Execució del projecte».}

\else
\chapter{Sustainability Analysis and Ethical Implications}

The sustainability of technological innovations, particularly those involving artificial intelligence and large-scale data processing, requires careful assessment across environmental, social, economic, and ethical dimensions. This section evaluates the broader impacts of the project and discusses potential societal implications associated with advancing this technology.

%Environmental Impact

The development, training, and deployment of machine learning models require substantial computational workloads, which translate into energy consumption and associated environmental costs. In this project, the processing of visual data during the preparation phase and the subsequent fine-tuning and evaluation of the model contribute to the overall computational demand.

%Resource Usage

Training machine learning models requires considerable electricity due to the intensive operations involved. The NVIDIA T4 available on Google Colab, used for this project, consumes approximately 70W, emphasizing the importance of efficient resource usage during training.

The T4 GPU was selected because GPUs are optimized for parallel computation, enabling them to conduct multiple operations simultaneously. This significantly reduces training time compared to CPU-only systems and makes GPUs particularly effective for large-scale datasets and complex neural network architectures. While GPUs consume more power than CPUs, the computational efficiency they provide justifies their use in scenarios where model training time and throughput are critical.

Storage requirements also contribute to environmental impact. The project requires approximately [Xbytes] of storage, primarily due to the datasets used in model training. Storage infrastructure has its own energy overhead: maintaining 1 terabyte of data in a data center typically consumes around 100 kWh annually. Consequently, managing [Xbytes] of data would result in an estimated annual energy consumption of [X kWh], excluding the additional energy needed for data processing and cooling. Implementing effective data management and pruning strategies is therefore essential to minimize environmental impact when handling large-scale datasets.

%Carbon Footprint

Training AI models contributes to carbon emissions, as the electricity used by computational hardware often originates from carbon-intensive energy sources. Although an exact carbon footprint cannot be calculated without detailed measurements of power usage, hardware efficiency, and local energy mixes, we can estimate the approximate environmental impact associated with the model training process. These estimates help highlight the importance of optimizing training pipelines, adopting energy-efficient hardware, and exploring low-carbon or renewable energy options where available.

\fi


%%%% CONCLUSIONS AND FUTURE WORK %%%%
% Conclusions chapter. Please replace "conclusions.tex" entirely with your own content written in the desired language.
\ifcase\doclanguage\or
\chapter{Conclusions i Línies Futures}
  \section{Conclusions}
    \begin{itemize}
      \item Resumiu els resultats principals del vostre treball.
      \item Discutiu el grau d'assoliment en relació amb els objectius marcats a l'inici del treball.
      \item Destaqueu les contribucions del vostre treball al camp d'estudi.
    \end{itemize}
  
  \section{Línies Futures}
    \begin{itemize}
      \item Identifiqueu àrees per a futures investigacions o desenvolupament basades en el vostre treball.
      \item Discutiu possibles vies per ampliar o millorar el projecte.
      \item Considereu les preguntes que han quedat sense resposta i les oportunitats per a futures exploracions.
    \end{itemize}

\else
\chapter{Conclusions and Future Work}

    The proposed pipeline demonstrates that the integration of distillation, low-rank adaptation, and quantization provides a robust framework for model compression. The findings confirm that it is possible to achieve a reduction in memory usage compared to the teacher model while maintaining competitive accuracy levels. This makes the resulting model suitable for resource-constrained environments where the original baseline would be computationally infeasible.
    
    Currently, the compression stages were executed with fixed parameters: a specific rank r for LoRA and a specific bit-width for quantization. Future research could investigate how to automatically find the optimal balance between distillation temperature and quantization thresholds.
    
    A further line of research is to frame the distillation process within an information-theoretic context and explore extending the pipeline to a Federated Learning setting for decentralized, privacy-preserving training.
    
    While this thesis focused on theoretical model size and accuracy, the pipeline could be extended by testing performance on diverse edge devices such as Raspberry Pi or mobile CPUs. This would provide empirical data on "Inference latency vs. power consumption" that goes beyond static model size.

\fi


%%%% BIBLIOGRAPHY %%%%
\nocite{*}               % Forces all entries to be printed, even if not cited
% Bibliography intro
% Supress this macro (or modify its contents to suit your needs).
\defbibnote{bib-intro}{%
\ifcase\doclanguage\or
  El sistema \textit{biblatex} simplifica la gestió de la bibliografia en treballs científics, proporcionant automatització i personalització en el format de les citacions. Això permet a l'autor del document enfocar-se en el contingut sense haver de preocupar-se per l'estil de les referències, estalviant temps i reduint errors.
  
  La base de dades de referències bibliogràfiques és al fitxer «TFG.bib» i és allà on heu d'afegir les vostres referències. Consulteu el manual del \texttt{biblatex}, secció «Database Guide», per conèixer els tipus de referències i camps disponibles.
  
  Podeu modificar (o suprimir) aquesta nota editant la macro \texttt{\textbackslash defbibnote} al fitxer «TFG.tex».
  \par\hfil\rule[3pt]{.5\textwidth}{0.4pt}\hfil\par\else
  %The \textit{biblatex} system simplifies the management of the bibliography in scientific works, providing automation and customization in the format of citations. This allows the document's author to focus on the content without having to worry about the style of the references, saving time and reducing errors.
  
  %The bibliographic references database is in the file “TFG.bib”, and this is where you should add your references. Consult the \texttt{biblatex} manual, section “Database Guide”, to learn about the available types of references and fields.
  
  %You can modify (or delete) this note by editing the \texttt{\textbackslash defbibnote} macro in the file “TFG.tex”.
  %\par\hfil\rule[3pt]{.5\textwidth}{0.4pt}\hfil\par\fi%
}
\printbibliography[heading=bibintoc,prenote={bib-intro}]


%%%% ANNEXES %%%%
% All chapters AFTER the \appendix command wil be considered appendices and numbered by letter
\appendix
\ifcase\doclanguage\or
  \chapter{Un apèndix}
  Es poden incloure apèndixs al TFG però no és obligatori.
\else
  \chapter{An appendix}
  Appendices may be included in your thesis but it is not a requirement.
\fi


%%%% FINAL CLOSING BLANK PAGE %%%%
\clearpage
\thispagestyle{empty}
~

\end{document}
